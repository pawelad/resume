% -- Encoding UTF-8 without BOM
% -- XeLaTeX => PDF (BIBER)

\documentclass[]{cv-style}  % Add 'print' as an option into the square bracket to remove colours from this template for printing.

\begin{document}

\header{pawel.ad}{amczak}
\lastupdated

%----------------------------------------------------------------------------------------
%	SIDEBAR SECTION  -- In the aside, each new line forces a line break
%----------------------------------------------------------------------------------------

\begin{aside}
%
\section{contact}
Paweł Adamczak
+48 732 005 766
~
Seweryna 3/49
30-632 Kraków
Poland
~
\href{https://pawelad.me/}{https://pawelad.me}
\href{mailto:pawel.ad@gmail.com}{gmail://pawel.ad}
\href{https://www.github.com/pawelad}{github://pawelad}
%
\section{languages}
native Polish
fluent English
%
\section{favorite tech}
{\color{red} $\varheartsuit$} Python
~
Django, black, Sentry, Go, Docker, Terraform, PostgreSQL, Redis, GitHub, Cloudflare, DevOps
\end{aside}

%----------------------------------------------------------------------------------------
%	ABOUT SECTION
%----------------------------------------------------------------------------------------

\section{about}
  \vspace{-0.3cm}

I'm a Pythonic developer that enjoys using \italica{and} creating software that can be called objectively good (or as close to it as possible). I believe good practices and attention to detail are very important in general and especially crucial in IT. I like tasks that are out of my comfort zone and the knowledge that comes with them. I think humility and open-mindedness are two of the most important qualities a person can have. My antiheroes are Bender Rodríguez and David Brent.

%----------------------------------------------------------------------------------------
%	WORK EXPERIENCE SECTION
%----------------------------------------------------------------------------------------

\section{experience}
  \vspace{-0.3cm}

\begin{entrylist}
%------------------------------------------------
\entry
  {08.18 -- Now}
  {Twenty3 Sport / All Sports Whispers}
  {Remote / London, UK}
  {Originally hired to help the data science team productise their code by converting PoC into MVPs, deploy them to AWS and implement best coding practices (i.e. CI / CD, Sentry, Docker based environments). I then worked on rewriting our football data digestion pipeline that imports and post-processes thousands of events each day (from multiple data providers) and exposes it via Django REST Framework based API. Created in 2019, it's still actively developed and empowers tools used by customers like \href{https://www.skysports.com/}{SkySports}, \href{https://www.squawka.com/}{Squawka}, \href{https://www.livescore.com/}{LiveScore} and \href{https://www.williamhill.com/}{William Hill}.}
%------------------------------------------------
\entry
  {09.17 -- 07.18}
  {VM Farms}
  {Toronto, Canada}
  {Part of a platform engineering team that maintained a custom build deployment pipeline, a client facing portal and an all purpose internal CLI. We worked directly with our in-house operations team to provide them with the tools to efficiently provision, configure and continuously support our client's servers.}
%------------------------------------------------
\entry
  {10.16 -- 08.17}
  {Omni Digital}
  {Bristol, UK}
  {Important member of an agile Python development team working on various projects - both internal and public facing - for clients like \href{https://www.bristolccg.nhs.uk/}{NHS}, \href{https://www.ukchamberofshipping.com/}{UK Chamber of Shipping} and \href{http://swns.com/}{SWNS}. I was also involved in implementing a company wide Ansible stack used for all our deployments.}
%------------------------------------------------
\entry
  {09.15 -- 09.16}
  {Sidnet Solutions / Pythonity}
  {Remote / Warsaw, Poland}
  {A one-man Python development team responsible for creating and maintaining Schedoodle, our internal tool that helped track availability of all employees, an official Polish site for Kolbe personality tests, and a canola trading platform made for one of the biggest Polish oil distributors. I was also tasked with releasing and maintaining some of our projects on \href{https://github.com/Pythonity}{GitHub}.}
%------------------------------------------------
\entry
  {04.15 -- 09.15}
  {JMR Technologies}
  {Warsaw, Poland}
  {Helped develop and release two of the company's biggest Python projects - a fully multilingual and customisable product knowledge base, and a parcel tracking status aggregator. I was also responsible for managing our production server and creating most of our provisioning automation scripts.}
%------------------------------------------------
\end{entrylist}

\vspace{-0.3cm}

%----------------------------------------------------------------------------------------
%	EDUCATION SECTION
%----------------------------------------------------------------------------------------

\section{education}
  \vspace{-0.3cm}

\begin{entrylist}
\entry
  {2013 -- 2016}
  {Warsaw University of Technology}
  {University}
  {}
\entry
  {2010 -- 2013}
  {I Liceum Ogólnokształcące w Zielonej Górze}
  {High School}
  {}
\end{entrylist}

\vspace{-0.5cm}

%----------------------------------------------------------------------------------------
%	OPEN SOURCE SECTION
%----------------------------------------------------------------------------------------

\section{open source}
  \vspace{-0.3cm}

\begin{itemize}
  \item \href{https://github.com/pawelad/pymonzo}{pymonzo} - Modern Python API client for Monzo public API.
  \item \href{https://github.com/pawelad/monz}{monz} - Simple CLI for your Monzo account.
  \item \href{https://github.com/pawelad/fakester}{fakester} - Rickroll your boss while preserving the element of surprise.
  \item Small contributions to open source projects: \href{https://github.com/pytest-dev/pytest}{pytest}, \href{https://github.com/tox-dev/tox}{tox}, \href{https://github.com/celery/celery}{celery}, \href{https://github.com/errbotio/errbot}{errbot}, \href{https://github.com/mattermost/mattermost}{mattermost}, \href{https://github.com/jazzband/django-polymorphic}{django-polymorphic}, \href{https://github.com/bennylope/django-organizations}{django-organizations} \href{https://github.com/etianen/django-reversion}{django-reversion}, \href{https://github.com/jgorset/django-kronos}{django-kronos}
\end{itemize}

%----------------------------------------------------------------------------------------
%	INTERESTS SECTION
%----------------------------------------------------------------------------------------

\section{interests}
  \vspace{-0.3cm}

traveling, \ \ \ python, \ \ \ nba, \ \ \ open source, \ \ \ photography, \ \ \ music making, \ \ \ tv, \ \ \ movies, \ \ \
%----------------------------------------------------------------------------------------

\vspace{-0.5cm}

\end{document}
